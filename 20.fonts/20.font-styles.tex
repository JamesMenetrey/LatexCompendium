\documentclass{article}

\usepackage[utf8]{inputenc}
\usepackage[T1]{fontenc}

% Required for underlines (the parameter normalem prevents ulem to redefine emph)
\usepackage[normalem]{ulem}

\newcommand{\lorem}{Lorem ipsum dolor sit amet, consectetur adipiscing elit.}

\begin{document}

\section*{Normal}

\begin{rm}
\lorem
\end{rm}

\section*{Boldface}

\textbf{\lorem}

{\bfseries \lorem}

\begin{bf}
\lorem
\end{bf}

\section*{Italic}

\textit{\lorem}

{\itshape \lorem}

\begin{it}
\lorem
\end{it}

\section*{Slanted}

\textsl{\lorem}

{\slshape \lorem}

\begin{sl}
\lorem
\end{sl}

\section*{Typewriter}

\texttt{\lorem}

{\ttfamily \lorem}

\begin{tt}
\lorem
\end{tt}

\section*{Sans serif}

\textsf{\lorem}

{\sffamily \lorem}

\begin{sf}
\lorem
\end{sf}

\section*{Small caps}

\textsc{\lorem}

{\scshape \lorem}

\begin{sc}
\lorem
\end{sc}

\section*{Exponent}

Hello\textsuperscript{world!}

\section*{Box}

\fbox{\lorem}

\section*{Underline}

\uline{\lorem}

\section*{Double underline}

\uuline{\lorem}

\section*{Underwave}

\uwave{\lorem}

\section*{Strikethrough}

\sout{\lorem}

\section*{Emphasize}

The markup \textit{\textbackslash emph} enables the editor to highlight words and behaves differently according the current style of the text. The next examples highlights the word \emph{Ipsum}:

\begin{itemize}
\item Lorem \emph{ipsum} dolor sit amet, consectetur adipiscing elit.
\item \textbf{Lorem \emph{ipsum} dolor sit amet, consectetur adipiscing elit.}
\item \textit{Lorem \emph{ipsum} dolor sit amet, consectetur adipiscing elit.}
\end{itemize}


	
\end{document}