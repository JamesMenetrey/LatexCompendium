% Thanks to Geoffrey Jones, author of this document, found at http://tex.stackexchange.com/a/5629/108649

% Note this document is two-sided
\documentclass[twoside]{article}

\usepackage{graphicx}
\usepackage[dvipsnames]{xcolor}
\usepackage{caption}[2008/04/01]
\usepackage{wrapfig}
\usepackage{subfig}

\usepackage{lipsum}  % to provide "filler" text

\captionsetup{
	justification=raggedright,
	labelfont={color=Maroon,bf},
	font=small}

\begin{document}
	\lipsum[1-2]
	
	% The commande marginpar enables the writer to specify content in margins.
	% Three marginpar figures to see how this looks in
	% the left and right margins of twoside docs...
	\marginpar{\includegraphics[width=0.95\marginparwidth]{resources/figure.png}
		\captionof{figure}{Carl Friedrich Gauss, født 1777, er bedst kendt for normalfordelingen}}
	\lipsum[1-2]
	\marginpar{\includegraphics[width=0.95\marginparwidth]{resources/figure.png}
		\captionof{figure}{Carl Friedrich Gauss, født 1777, er bedst kendt for normalfordelingen}}
	\lipsum[1-4]
	\marginpar{\includegraphics[width=0.95\marginparwidth]{resources/figure.png}
		\captionof{figure}{Carl Friedrich Gauss, født 1777, er bedst kendt for normalfordelingen}}
	\lipsum[5-6]
	
	\lipsum[2]
	\setlength\columnsep{\marginparsep}
	
	% The environment wrapfigure enables to insert a content next to a portion of text.
	% The prototype of the environment is:
	%	{wrapfigure}
	%		[number of lines required to properly integrate the side content]
	%		{alignment of the content}
	%		{size of the allowable overflow in the margin}
	%		[width of the content]
	
	\begin{wrapfigure}[33]{r}[3cm]{0.46\textwidth}
		\vspace{-1.5\baselineskip}
		\subfloat[Giveaway pr. beholder i procent]{\label{lbl1}
			\includegraphics[width=0.45\textwidth]{resources/figure.png}}\par
		\subfloat[Giveaway pr. beholder i procent]{\label{lbl2}
			\includegraphics[width=0.45\textwidth]{resources/figure.png}}
		\caption{Resultater for testsæt 1 blah blah blah blah blah blah blah.}
	\end{wrapfigure}
	\lipsum[11-14]
	
\end{document}