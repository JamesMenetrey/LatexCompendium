\documentclass{article}

\usepackage[utf8]{inputenc}
\usepackage[T1]{fontenc}

% Main maths packages
\usepackage{amsmath}
\usepackage{amssymb}
\usepackage{mathrsfs}

% Declare a differential command in order to not display the "d" in italic
\newcommand*{\dd}[1][x]{\,\mathrm{d}#1}

\begin{document}

\section*{Integrals}

\subsection*{Simple integrals}

\[\int x^2 \dd\]

% The bounds are written at the right of the integral
\[\int_{0}^{\infty} x^2 \dd\]

% The bounds are written at the top and bottom of the integral by specifying \limits.
%Refers styling.tex for more information about the command \limits.
\[\int\limits_{0}^{\infty} x^2 \dd\]

\[\oint x^2 \dd\]

\subsection*{Double integrals}

\[\iint x^2 \dd \]

\[\int\int x^2 \dd\]

\[\int_{0}^{4}\int_{1}^{3} x^2 \dd\]

\subsection*{Triple integrals}

\[\iiint x^2 \dd\]

\[\int_{x=0}^{x=5}\int_{y=0}^{y=4}\int_{z=1}^{z=3} x^2+y-z \dd\]

\[\int\limits_{x=0}^{x=5}\int\limits_{y=0}^{y=4}\int\limits_{z=1}^{z=3} x^2+y-z \dd\]

\end{document}