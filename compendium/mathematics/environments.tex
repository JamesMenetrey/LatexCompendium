\documentclass{article}

\usepackage[utf8]{inputenc}
\usepackage[T1]{fontenc}

% Set the scale of the height of the rows (set to 1.5 relative to its default height)
\renewcommand{\arraystretch}{2}

\begin{document}

\section*{Inline (within text) formulas}

The equation $x + x = 2x$ is inside a text, which uses the \TeX{} shorthand.
In addition, an equation can be written like \(x \cdot x = x^2\) as well, using the \LaTeX{} shorthand.
Finally, the environment \emph{math} can be used like this \begin{math}a^2 + b + c = 0\end{math}.

We can force symbols to be displayed like displayed formula. For example the formula $\sum_{k=0}^{10}k$ can be written $\displaystyle\sum_{k=0}^{10}k$ as well. The sum symbol is taller within a sentence using the command \emph{\textbackslash displaystyle}.

\section*{Displayed equations}

The recommended syntax to render a floating equation is to use the syntax below.

\[a \cdot x = ax \]

The use of the syntax \emph{\$\$$\cdots$\$\$} should be avoided, because, it will modify vertical spacing within equations, rendering them inconsistent.

Finally, the environment \emph{displaymath} produces the same effect.

\begin{displaymath}
	x^2 \cdot x^2 = x^4
\end{displaymath}

\section*{Style of equations}

Several symbols such as the sum or an integral can be rendered differently, depending on whether this is an inline or displayed formula. We can force the rendering of the different styles with some specific commands. The table \ref{table:custom-rendering} summaries the commands with a concrete examples.

\begin{table}[h]
	\centering
	\begin{tabular}{| c | c | c |}
		\hline
		& \textbackslash displaystyle & \textbackslash textstyle \\
		\hline
		\textbackslash nolimits	& $\displaystyle\sum\nolimits_{k=0}^{n}k$	& $\textstyle\sum\nolimits_{k=0}^{n}k$ \\
		\hline
		\textbackslash limits	& $\displaystyle\sum\limits_{k=0}^{n}k$	& $\textstyle\sum\limits_{k=0}^{n}k$ \\
		\hline
	\end{tabular}
	\caption{Summary of commands and styling}
	\label{table:custom-rendering}
\end{table}

Other styling commands can be applied instead of \emph{displaystyle} or \emph{textstyle}, such as \emph{scriptstyle} or \emph{scriptscriptstyle}.

\end{document}