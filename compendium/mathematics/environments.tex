\documentclass{article}

\usepackage[utf8]{inputenc}
\usepackage[T1]{fontenc}

\begin{document}

\section*{Inline (within text) formulas}

The equation $x + x = 2x$ is inside a text, which uses the \TeX{} shorthand.
In addition, an equation can be written like \(x \cdot x = x^2\) as well, using the \LaTeX{} shorthand.
Finally, the environment \emph{math} can be used like this \begin{math}a^2 + b + c = 0\end{math}.

We can force symbols to be displayed like displayed formula. For example the formula $\sum_{k=0}^{10}k$ can be written $\displaystyle\sum_{k=0}^{10}k$ as well. The sum symbol is taller within a sentence using the command \emph{\textbackslash displaystyle}.

\section*{Displayed equations}

The recommended syntax to render a floating equation is to use the syntax below.

\[a \cdot x = ax \]

The use of the syntax \emph{\$\$$\cdots$\$\$} should be avoided, because, it will modify vertical spacing within equations, rendering them inconsistent.

Finally, the environment \emph{displaymath} produces the same effect.

\begin{displaymath}
	x^2 \cdot x^2 = x^4
\end{displaymath}

\section*{Equation numbering}

The environment \emph{equation} automatically numbers the equations.

\begin{equation} 
	f(x)=(x+a)(x+b)
\end{equation}

\end{document}