\documentclass{article}

\usepackage[utf8]{inputenc}
\usepackage[T1]{fontenc}

% Main maths packages
\usepackage{amsmath}
\usepackage{amssymb}
\usepackage{mathrsfs}

\usepackage{tabularx}

% First and foremost, never use the environment eqnarray. It is not recommended because spacing is inconsistent.

\begin{document}

\section*{System of equations}


\subsection*{Summary of the environments}

\begin{table}[h]
	\begin{tabularx}{\textwidth}{| X | X | X |}
	\hline
	\textbf{Environment names} & \textbf{Description} & \textbf{Notes} \\
	\hline
	\emph{gather} and \emph{gather*} & Consecutive equations with an alignment to the center. & \\
	\hline
	\emph{align}, \emph{align*}, \emph{aligned} and \emph{split} & Consecutive equations with user-defined alignment. & \\
	\hline
	\emph{flalign} and \emph{flalign*} & Similar to \emph{align}, but left aligns first equation column, and right aligns last column. & \\
	\hline
	\emph{alignat} and \emph{alignat*} & Takes an argument specifying number of columns. Allows control of the horizontal space between equations. & This environment takes one argument, the number of “equation columns”: count the maximum number of \&s in any row, add 1 and divide by 2. \\
	\hline
	\emph{array} & Advanced alignment scenario. Enables to control how to columns are aligned. & \\
	\hline
	\emph{multline} and \emph{multline*} & First line left aligned, last line right aligned. & Equation number aligned vertically with first line and not centered as with other environments. \\
	\hline
	\emph{eqnarray} and \emph{eqnarray*} & Similar to \emph{align} and \emph{align*}. & Not recommended because spacing is inconsistent. \\
	\hline
	\end{tabularx}
\end{table}

This table is based on the one available on Wikibooks, \LaTeX Advanced Mathematics.

\newpage


\subsection*{Grouping and centering equations with \emph{gather}}

\begin{gather}
	x + 2y - z = 4 \\
	x + y - 5z = -1 \\
	2x - z = 10
\end{gather}

In addition, the environment \emph{gather*} removes the equation numbering. The environment \emph{gathered} and can be used in another mathematical environment.


\subsection*{Align equations with \emph{align}}

\begin{align}
	x + 2y - z &= 4 \\
	x + y - 5z &= -1 \\
	2x - z &= 10
\end{align}

This environment can also be used to align equations on the same line:

\begin{align*}
	f(x)  &= a x^2+b x +c   &   g(x)  &= d x^3 \\
	f'(x) &= 2 a x +b       &   g'(x) &= 3 d x^2
\end{align*}

In addition, the environment \emph{align*} removes the equation numbering. The environment \emph{aligned} and \emph{split} are the similar to \emph{align} and \emph{align*} but can be used in another mathematical environment.

\subsubsection*{Align equations with \emph{align} and \emph{subequations}}

The environment \emph{subequations} changes the way how \LaTeX counts the equations.

\begin{subequations}
	\begin{align}
		x + 2y - z &= 4 \\
		x + y - 5z &= -1 \\
		2x - z &= 10
	\end{align}
\end{subequations}

\subsubsection*{Braces a system of equations}

The environment \emph{aligned} is similar to align, to be used inside another mathematics environment.

% The commands \left and \right insert a brace. The type of the brace is written just after the command.
% When \left is called, \right is mandatory.
\[
\left(
\begin{aligned}
	x + 2y - z &= 4 \\
	x + y - 5z &= -1 \\
	2x - z &= 10
\end{aligned}
\right)
\]

% In the case where only one brace is needed, the other brace must be marked with a point (.) to hide it.
\[
\left\{
\begin{aligned}
x + 2y - z &= 4 \\
x + y - 5z &= -1 \\
2x - z &= 10
\end{aligned}
\right.
\]


\subsection*{Align equations to the left with \emph{flalign}}

\begin{flalign}
a &= b+c &\\
&= 1+1 &\\
&= 2  &
\end{flalign}

In addition, the environment \emph{flalign*} removes the equation numbering.


\subsection*{Align equations with \emph{alignat}}

The environment \emph{alignat} enables to control of the horizontal space between equations. Indeed, no addition space is added between equation, at the opposite of the environment \emph{align}.

The mathematical development below is aligned according the arrow and the equal signs.

\begin{alignat}{2}
	&& \sqrt{4 x^2} - 1 &= 0 \\
	&\Rightarrow & \sqrt{4} \sqrt{x^2} &= 1 \\
	&\Rightarrow & 2 \left|x\right| &= 1 \\
	&\Rightarrow & x &= \pm \frac{1}{2}
\end{alignat}

This environment takes an argument specifying number of columns. The rule of thumb to determine the argument is to count the maximum number of \& symbols on one row, add 1 and divide by 2.

In addition, the environment \emph{alignat*} removes the equation numbering. The environment \emph{alignedat} and \emph{alignedat*} and can be used in another mathematical environment.


\subsection*{Align equations with \emph{array}}

The environment \emph{array} is for more advanced scenario. It is basically the same as \emph{align}, but the columns and their alignment are explicitly indicated. It must be used in the math mode.

\[
\left\{
\begin{array}{ r c l}
		x + z &=& y - 4 \\
		5z &=& x + y -1 \\
		2x - y &=& 10 - y
\end{array}
\right.
\]


\subsection*{Displaying long equations}

The environment \emph{multline} can be used to display formula on multiple lines.

\begin{multline*}
	f(x) = 60x^{15} + 56x^{14} + 52x^{13} + 48x^{12} + 44x^{11} + 40x^{10} + 36x^{9} + 32x^{8}\\
	+ 28x^{7} + 24x^{6} + 20x^{5} + 16x^{4} + 12x^{3} + 8x^{2} + 4
\end{multline*}

\end{document}