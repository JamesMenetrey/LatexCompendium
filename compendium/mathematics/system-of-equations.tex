\documentclass{article}

\usepackage[utf8]{inputenc}
\usepackage[T1]{fontenc}

% Main maths packages
\usepackage{amsmath}
\usepackage{amssymb}
\usepackage{mathrsfs}

% First and foremost, never use the environment eqnarray. It is not recommended because spacing is inconsistent.

\begin{document}

\section*{System of equations}

\subsection*{Grouping and centering equations with \emph{gather} (numbered)}

\begin{gather}
	x + 2y - z = 4 \\
	x + y - 5z = -1 \\
	2x - z = 10
\end{gather}

In addition, he environment \emph{gathered} can be used another mathematical environment.

\subsection*{Grouping and centering equations with \emph{gather} (unnumbered)}

\begin{gather*}
x + 2y - z = 4 \\
x + y - 5z = -1 \\
2x - z = 10
\end{gather*}

\subsection*{System with \emph{align} (numbered)}

\begin{align}
	x + 2y - z &= 4 \\
	x + y - 5z &= -1 \\
	2x - z &= 10
\end{align}

\subsection*{System with \emph{align} (numbered) and \emph{subequations}}

\begin{subequations}
	\begin{align}
		x + 2y - z &= 4 \\
		x + y - 5z &= -1 \\
		2x - z &= 10
	\end{align}
\end{subequations}

\subsection*{System with \emph{align*} (unnumbered)}

\begin{align*}
	x + 2y - z &= 4 \\
	x + y - 5z &= -1 \\
	2x - z &= 10
\end{align*}

This environment can also be used to align equations on the same line:

\begin{align*}
	 f(x)  &= a x^2+b x +c   &   g(x)  &= d x^3 \\
	 f'(x) &= 2 a x +b       &   g'(x) &= 3 d x^2
\end{align*}

\subsection*{Braces with \emph{aligned}}

The environment \emph{aligned} is similar to align, to be used inside another mathematics environment.

% The commands \left and \right insert a brace. The type of the brace is written just after the command.
% When \left is called, \right is mandatory.
\[
\left(
\begin{aligned}
	x + 2y - z &= 4 \\
	x + y - 5z &= -1 \\
	2x - z &= 10
\end{aligned}
\right)
\]

% In the case where only one brace is needed, the other brace must be marked with a point (.) to hide it.
\[
\left\{
\begin{aligned}
x + 2y - z &= 4 \\
x + y - 5z &= -1 \\
2x - z &= 10
\end{aligned}
\right.
\]

\subsection*{System with \emph{array}}

The environment \emph{array} is for more advanced scenario. It is basically the same as \emph{align}, but the columns and their alignment are explicitly indicated. It must be used in the math mode.

\[
\left\{
\begin{array}{ r c l}
		x + z &=& y - 4 \\
		5z &=& x + y -1 \\
		2x - y &=& 10 - y
\end{array}
\right.
\]

\subsection*{Displaying long equations}

The environment \emph{multline} can be used to display formula on multiple lines.

\begin{multline*}
	f(x) = 60x^{15} + 56x^{14} + 52x^{13} + 48x^{12} + 44x^{11} + 40x^{10} + 36x^{9} + 32x^{8}\\
	+ 28x^{7} + 24x^{6} + 20x^{5} + 16x^{4} + 12x^{3} + 8x^{2} + 4
\end{multline*}

\end{document}