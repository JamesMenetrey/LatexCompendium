\documentclass{article}

\usepackage[utf8]{inputenc}
\usepackage[T1]{fontenc}

% Main maths packages
\usepackage{amsmath}
\usepackage{amssymb}
\usepackage{mathrsfs}
\usepackage{mathtools}

\begin{document}

\section*{Piecewise functions}

\subsection*{The environment \emph{cases}}

The best practices to define piecewise functions is to the environment \emph{cases}.

\[
	\operatorname{abs}(x)=
	\begin{cases}
		-x	& \text{if } x < 0 \\
		x	& \text{if } x \ge 0
	\end{cases}
\]

\subsection*{The manual way using \emph{aligned}}

Piecewise functions can also be done using a brace and the environment \emph{aligned}.

\[
	\operatorname{abs}(x)=
	\left\{
	\begin{aligned}
		-x	& : x < 0 \\
		x	& : x \ge 0
	\end{aligned}
	\right.
\]

Within cases, \emph{text} style math is used with results such as:

\[
	f(x)=
	\begin{cases}
		\int_{0}^{10} x\, dx & \text{if } x \ge 0 \\
		\frac{1}{x} & \text{if } x < 0
	\end{cases}
\]

\emph{Display} style may be used instead, by using the environment \emph{dcases} with the package \emph{mathtools}:

\[
	f(x)=
	\begin{dcases}
		\int_{0}^{10} x\, dx & \text{if } x \ge 0 \\
		\frac{1}{x} & \text{if } x < 0
	\end{dcases}
\]

Often the second column consists mostly of normal text. To set it in the normal font of the document, the \emph{dcases*} environment may be used:

\[
	f(x)=
	\begin{dcases*}
		\int_{0}^{10} x\, dx & when $x$ is even \\
		\frac{1}{x} & when $x$ is odd
	\end{dcases*}
\]

\subsection*{The manual way using \emph{array}}

Piecewise functions can also be done using a brace and the environment \emph{array}. This enables to control the alignment of the elements.

\[
	\operatorname{abs}(x)=
	\left\{
	\begin{array}{lr}
		-x	& : x < 0 \\
		x	& : x \ge 0
	\end{array}
	\right.
\]

\end{document}