\documentclass{article}

\usepackage[utf8]{inputenc}
\usepackage[T1]{fontenc}

% Required to use \cfrac
\usepackage{amsmath}

\begin{document}

\section*{Simple fraction}

Fractions are created as follows.

% Syntax: \frac{numerator}{denominator}

\[\frac{1}{x} = x^{-1}\]

\section*{Continued (or cascade) fractions}

A fraction can contain another one. If the command \emph{\textbackslash frac} is, this would be the result:

\[\frac{x}{x + \frac{1}{2}}\]

The sub fraction is very small. This is why the command \emph{\textbackslash cfrac} is recommended for continued fractions:

\[\cfrac{x}{x + \cfrac{1}{2}}\]

Some people prefer to still use \emph{\textbackslash frac} and use the command \emph{\textbackslash displaystyle} before each sub fractions:

\[\frac{x}{x + \displaystyle\frac{1}{2}}\]

Despite this practice is sometimes promoted, it should be avoided because the equation has some spacing issues, such as on top of $\frac{1}{2}$.

\end{document}