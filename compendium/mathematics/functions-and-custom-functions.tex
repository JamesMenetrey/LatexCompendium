\documentclass{article}

\usepackage[utf8]{inputenc}
\usepackage[T1]{fontenc}

% Required to call the command \operatorname{name} and \DeclareMathOperator{command}{definition}
\usepackage{amsmath}

% Required to render the clickable url
\usepackage[hyphens]{url}
\usepackage{hyperref}

% Declare a new math function
\DeclareMathOperator{\abc}{abc}

\begin{document}
	
\section*{Existing functions}

\LaTeX{} comes with a lot of built-in mathematical functions, such as $\cos x$, $\ln x$ or $\max x$.
A complete list of symbols can be found in the \emph{The Comprehensive LATEX Symbol List} hosted in the \emph{Best Practices} part of this repository (\url{https://github.com/ZenLulz/LatexCompendium/blob/master/best-practices/the-comprehensive-latex-symbol-list.pdf}).

\section*{Custom functions}
\subsection*{The command \emph{operatorname}}

The following formula uses a custom function called \emph{abc}.

\[\operatorname{abc} x\]

\subsection*{The command \emph{DeclareMathOperator}}

This command enables to define math functions or operators in the preamble so they can be reused everywhere.

\[\abc x\]

\subsection*{The command \emph{mathrm}}

Many people use the command \emph{mathrm}, nevertheless this leads to space issue, as illustrated below.

\[\mathrm{abc}x\]

\end{document}