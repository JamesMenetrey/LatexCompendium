\documentclass{article}

\usepackage[utf8]{inputenc}
\usepackage[T1]{fontenc}

% Required to use the commands >{}.
\usepackage{array}

% Required to use colors and alternate colors (all the features of the package "color" are available when loading the package "xcolor").
\usepackage[table]{xcolor}
\definecolor{lightgray}{gray}{0.9}
\definecolor{normalgray}{gray}{0.8}
\definecolor{darkgray}{gray}{0.5}

% Required to apply colors to cell/row/column.
\usepackage{colortbl}

% Define the color of the borders of the table
\arrayrulecolor{darkgray}

% Set the thickness of the borders
\setlength{\arrayrulewidth}{1pt}

% Set the space between the content of the cell and the left/right border.
\setlength{\tabcolsep}{10pt}

% Set the scale of the height of the rows (set to 1.5 relative to its default height)
\renewcommand{\arraystretch}{1.5}

\begin{document}

\begin{table}
	\centering
	% The command >{} executes has been discovered in the reference apply-commands-on-columns.tex.
	% The command \columncolor{color} colors a whole column at once.
	\begin{tabular}{| >{\columncolor{lightgray}} c | c | c | c |}
		\hline
		% The command \rowcolor{color} colors a whole row at once.
		\rowcolor{lightgray}
		$\times$ & 1 & 2 & 3 \\
		\hline
		1 & 1 & 2 & 3 \\
		\hline
		2 & 2 & 4 & 6 \\
		\hline
		% The command \cellcolor{color} colors a particular cell.
		3 & 3 & 6 & \cellcolor{lightgray} 9 \\
		\hline
	\end{tabular}
	\caption{Table with some colors}
\end{table}

\begin{table}
	% The command \rowcolor[color model]{color}[left overhang][right overhang] is used to display alternate colors.
	\rowcolors{2}{lightgray}{normalgray}
	\centering
	\begin{tabular}{| l l |}
		\hline
		\textbf{Country} & \textbf{Capital} \\
		\hline
		France & Paris \\
		\hline
		Germany & Berlin \\
		\hline
		Switzerland & Bern \\
		\hline
		United Kingdom & London \\
		\hline
	\end{tabular}
	\caption{Table with alternate colors}
\end{table}

\end{document}