\documentclass{article}

\usepackage[utf8]{inputenc}
\usepackage[T1]{fontenc}

\usepackage{multirow}

\begin{document}
	
	\begin{table}
		\centering
		\begin{tabular}{| c | c | c |}
			\hline
			% The command \multirow{number of rows}{width}{text} merges rows. The parameter width is optional and can be replaced by an asterisk (*).
			\multirow{2}*{1} & 2 & 3 \\
			% The command \cline{i-j} draws a horizontal border between the cell i and cell j (counted from left to right, starting from 1).
			\cline{2-3}
			% When two rows are merged, the subsequent rows must reserve an empty cell for the merged cell.
			& \multirow{2}*{5} & 6 \\
			\cline{1-1} \cline{3-3}
			4 & & \multirow{2}*{9} \\
			\cline{1-2}
			7 & 8 & \\
			\hline
		\end{tabular}
		\caption{Table with merged rows}
	\end{table}
	
\end{document}