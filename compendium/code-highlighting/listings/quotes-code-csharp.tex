\documentclass{article}

\usepackage[utf8]{inputenc}
\usepackage[T1]{fontenc}
\usepackage{hyperref}

% Set the font of the code listing
\usepackage{inconsolata} % other fonts can be used such as \usepackage{courier}

% Define the colors used in the code listing
\usepackage{color}
\definecolor{bluekeywords}{rgb}{0,0,1}
\definecolor{greencomments}{rgb}{0,0.5,0}
\definecolor{redstrings}{rgb}{0.64,0.08,0.08}
\definecolor{xmlcomments}{rgb}{0.5,0.5,0.5}
\definecolor{types}{rgb}{0.17,0.57,0.68}

% Set up the code listing
\usepackage{listings}
\lstset{
	% Programming language used
	language=[Sharp]C,
	% Caption at the bottom
	captionpos=b,
	% Display the line numbers
	numbers=left,
	% how far the line-numbers are from the code
	numbersep=10pt,
	% Style for the line numbers
	numberstyle=\tiny,
	% Use line as separator
	frame=lines,
	% Display the spaces in the listing
	showspaces=false,
	% Display the tabs in the listing
	showtabs=false,
	% Add break lines if needed (for lines too long)
	breaklines=true,
	% Display the spaces in the code strings
	showstringspaces=false,
	% sets if automatic breaks should only happen at whitespace
	breakatwhitespace=true,
	% if you want to add LaTeX within your code
	escapeinside={(*@}{@*)},
	% Set the color of the comments
	commentstyle=\color{greencomments},
	% Enchance the list of language keywords
	morekeywords={partial, var, value, get, set},
	% Set the color of the keywords
	keywordstyle=\color{bluekeywords},
	% Set the color of the strings
	stringstyle=\color{redstrings},
	% Set the style of the code. (try \ttfamily\small as well)
	basicstyle=\footnotesize\ttfamily,
	% sets default tabsize to 2 spaces
	tabsize=5,
}

\begin{document}
	
	
\begin{lstlisting}[label=label-here,caption=Assembler implementation - MemorySharp]
/*
* MemorySharp Library
* http://www.binarysharp.com/
*
* Copyright (C) 2012-2014 James Menetrey (a.k.a. ZenLulz).
* This library is released under the MIT License.
* See the file LICENSE for more information.
*/

using Binarysharp.Assemblers.Fasm;
using System;

namespace Binarysharp.MemoryManagement.Assembly.Assembler
{
	/// <summary>
	/// Implement Fasm.NET compiler for 32-bit development.
	/// More info: https://github.com/ZenLulz/Fasm.NET
	/// </summary>
	public class Fasm32Assembler : IAssembler
	{
		/// <summary>
		/// Assemble the specified assembly code.
		/// </summary>
		/// <param name="asm">The assembly code.</param>
		/// <returns>An array of bytes containing the assembly code.</returns>
		public byte[] Assemble(string asm)
		{
			// Assemble and return the code
			return Assemble(asm, IntPtr.Zero);
		}
		
		/// <summary>
		/// Assemble the specified assembly code at a base address.
		/// </summary>
		/// <param name="asm">The assembly code.</param>
		/// <param name="baseAddress">The address where the code is rebased.</param>
		/// <returns>An array of bytes containing the assembly code.</returns>
		public byte[] Assemble(string asm, IntPtr baseAddress)
		{
			// Rebase the code
			asm = String.Format("use32\norg 0x{0:X8}\n", baseAddress.ToInt64()) + asm;
			// Assemble and return the code
			return FasmNet.Assemble(asm);
		}
	}
}
\end{lstlisting}

\section*{Sources}

The style for this listing has been inspired by this StackExchange post: \url{http://tex.stackexchange.com/a/124969/108649}. \\

More information about this package can be found on \url{https://en.wikibooks.org/wiki/LaTeX/Source_Code_Listings}.
	
\end{document}