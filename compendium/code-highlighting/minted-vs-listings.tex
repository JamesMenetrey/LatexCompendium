\documentclass[a4paper]{article}

\usepackage[utf8]{inputenc}
\usepackage[T1]{fontenc}

\usepackage[dvipsnames]{xcolor}
\definecolor{bluekeywords}{rgb}{0,0,1}
\definecolor{greencomments}{rgb}{0,0.5,0}
\definecolor{redstrings}{rgb}{0.64,0.08,0.08}
\definecolor{xmlcomments}{rgb}{0.5,0.5,0.5}
\definecolor{types}{rgb}{0.17,0.57,0.68}

\usepackage{minted}
\usemintedstyle{pastie}

\usepackage{listings}
\lstset{
	% Programming language used
	language=[Sharp]C,
	% Caption at the bottom
	captionpos=b,
	% Display the line numbers
	numbers=left,
	% how far the line-numbers are from the code
	numbersep=10pt,
	% Style for the line numbers
	numberstyle=\tiny,
	% Use line as separator
	frame=lines,
	% Display the spaces in the listing
	showspaces=false,
	% Display the tabs in the listing
	showtabs=false,
	% Add break lines if needed (for lines too long)
	breaklines=true,
	% Display the spaces in the code strings
	showstringspaces=false,
	% sets if automatic breaks should only happen at whitespace
	breakatwhitespace=true,
	% if you want to add LaTeX within your code
	escapeinside={(*@}{@*)},
	% Set the color of the comments
	commentstyle=\color{greencomments},
	% Enchance the list of language keywords
	morekeywords={partial, var, value, get, set},
	% Set the color of the keywords
	keywordstyle=\color{bluekeywords},
	% Set the color of the strings
	stringstyle=\color{redstrings},
	% Set the style of the code. (try \ttfamily\small as well)
	basicstyle=\footnotesize\ttfamily,
	% sets default tabsize to 2 spaces
	tabsize=5,
}

\begin{document}

\section*{An SQL query using \emph{minted}}
\begin{minted}[linenos]{sql}
CREATE OR REPLACE TRIGGER Tel_On_Off
AFTER INSERT ON STATE_CHANGE
FOR EACH ROW
WHEN (NEW.ChangeType='O' OR NEW.ChangeType='F')
DECLARE
  CELLA NUMBER;
  N_TEL_ATTIVI_MAX NUMBER;
BEGIN
  --Trovo l'ID della cella in cui si trova il mio cellulare
  SELECT CellID, MaxCalls INTO CELLA, N_TEL_ATTIVI_MAX
  FROM CELL
  WHERE x0<=:NEW.x AND x1>=:NEW.x AND
        y0<=:NEW.y AND y1>=:NEW.y;
        
  --Telefono acceso
  IF(:NEW.ChangeType='O') THEN
    --Inserisco il nuovo cellulare nella tabella
    INSERT INTO TELEPHONE(PhoneNo, x, y, PhoneState)
    VALUES(:NEW.PhoneNo, :NEW.x, :NEW.y, :NEW.ChangeType);
    --Aggiorno la cella in cui si trova il cellulare
    UPDATE CELL SET CurrentPhone#=CurrentPhone#+1
    WHERE CellID=CELLA;
  END IF;
  
  --Telefono spento
  IF(:NEW.ChangeType='F') THEN
    --Rimozione telefono da tebella
    DELETE FROM TELEPHONE WHERE PhoneNo=:NEW.PhoneNo;
    --Aggiorno la cella in cui si trova il cellulare
    UPDATE CELL SET CurrentPhone#=CurrentPhone#-1
    WHERE CellID=CELLA;
  END IF; 
END;
\end{minted}

\newpage
\section*{An SQL query using \emph{listings}}
\begin{lstlisting}[
language=SQL,
showspaces=false,
basicstyle=\ttfamily,
numbers=left,
numberstyle=\tiny,
commentstyle=\color{gray}
]
CREATE OR REPLACE TRIGGER Tel_On_Off
AFTER INSERT ON STATE_CHANGE
FOR EACH ROW
WHEN (NEW.ChangeType='O' OR NEW.ChangeType='F')
DECLARE
  CELLA NUMBER;
  N_TEL_ATTIVI_MAX NUMBER;
BEGIN
  --Trovo l'ID della cella in cui si trova il mio cellulare
  SELECT CellID, MaxCalls INTO CELLA, N_TEL_ATTIVI_MAX
  FROM CELL
  WHERE x0<=:NEW.x AND x1>=:NEW.x AND
        y0<=:NEW.y AND y1>=:NEW.y;
        
  --Telefono acceso
  IF(:NEW.ChangeType='O') THEN
    --Inserisco il nuovo cellulare nella tabella
    INSERT INTO TELEPHONE(PhoneNo, x, y, PhoneState)
    VALUES(:NEW.PhoneNo, :NEW.x, :NEW.y, :NEW.ChangeType);
    --Aggiorno la cella in cui si trova il cellulare
    UPDATE CELL SET CurrentPhone#=CurrentPhone#+1
    WHERE CellID=CELLA;
  END IF;
  
  --Telefono spento
  IF(:NEW.ChangeType='F') THEN
    --Rimozione telefono da tebella
    DELETE FROM TELEPHONE WHERE PhoneNo=:NEW.PhoneNo;
    --Aggiorno la cella in cui si trova il cellulare
    UPDATE CELL SET CurrentPhone#=CurrentPhone#-1
    WHERE CellID=CELLA;
  END IF; 
END;
\end{lstlisting}

\newpage
\section*{A C\# chunk of code using \emph{minted}}
\usemintedstyle{vs}
\begin{minted}[linenos,breaklines]{csharp}
using System;
using System.Runtime.InteropServices;

namespace Binarysharp.MemoryManagement.Memory
{
    /// <summary>
    /// Class representing a block of memory allocated in the local process.
    /// </summary>
    public class LocalUnmanagedMemory : IDisposable
    {
        #region Properties
        /// <summary>
        /// The address where the data is allocated.
        /// </summary>
        public IntPtr Address { get; private set; }
        /// <summary>
        /// The size of the allocated memory.
        /// </summary>
        public int Size { get; private set; }
        #endregion

        #region Constructor/Destructor
        /// <summary>
        /// Initializes a new instance of the <see cref="LocalUnmanagedMemory"/> class, allocating a block of memory in the local process.
        /// </summary>
        /// <param name="size">The size to allocate.</param>
        public LocalUnmanagedMemory(int size)
        {
            // Allocate the memory
            Size = size;
            Address = Marshal.AllocHGlobal(Size);
        }
        /// <summary>
        /// Frees resources and perform other cleanup operations before it is reclaimed by garbage collection.
        /// </summary>
        ~LocalUnmanagedMemory()
        {
            Dispose();
        }
        #endregion

        #region Methods
        #region Dispose (implementation of IDisposable)
        /// <summary>
        /// Releases the memory held by the <see cref="LocalUnmanagedMemory"/> object.
        /// </summary>
        public virtual void Dispose()
        {
            // Free the allocated memory
            Marshal.FreeHGlobal(Address);
            // Remove the pointer
            Address = IntPtr.Zero;
            // Avoid the finalizer
            GC.SuppressFinalize(this);
        }
        #endregion
        #region Read
        /// <summary>
        /// Reads data from the unmanaged block of memory.
        /// </summary>
        /// <typeparam name="T">The type of data to return.</typeparam>
        /// <returns>The return value is the block of memory casted in the specified type.</returns>
        public T Read<T>()
        {
            // Marshal data from the block of memory to a new allocated managed object
            return (T)Marshal.PtrToStructure(Address, typeof(T));
        }
        /// <summary>
        /// Reads an array of bytes from the unmanaged block of memory.
        /// </summary>
        /// <returns>The return value is the block of memory.</returns>
        public byte[] Read()
        {
            // Allocate an array to store data
            var bytes = new byte[Size];
            // Copy the block of memory to the array
            Marshal.Copy(Address, bytes, 0, Size);
            // Return the array
            return bytes;
        }
        #endregion
    }
}
\end{minted}

\newpage
\section*{A C\# chunk of code using \emph{listings}}
\begin{lstlisting}[
language=C,
showspaces=false,
basicstyle=\ttfamily,
numbers=left,
numberstyle=\tiny,
breaklines=true]
using System;
using System.Runtime.InteropServices;

namespace Binarysharp.MemoryManagement.Memory
{
    /// <summary>
    /// Class representing a block of memory allocated in the local process.
    /// </summary>
    public class LocalUnmanagedMemory : IDisposable
    {
        #region Properties
        /// <summary>
        /// The address where the data is allocated.
        /// </summary>
        public IntPtr Address { get; private set; }
        /// <summary>
        /// The size of the allocated memory.
        /// </summary>
        public int Size { get; private set; }
        #endregion

        #region Constructor/Destructor
        /// <summary>
        /// Initializes a new instance of the <see cref="LocalUnmanagedMemory"/> class, allocating a block of memory in the local process.
        /// </summary>
        /// <param name="size">The size to allocate.</param>
        public LocalUnmanagedMemory(int size)
        {
            // Allocate the memory
            Size = size;
            Address = Marshal.AllocHGlobal(Size);
        }
        /// <summary>
        /// Frees resources and perform other cleanup operations before it is reclaimed by garbage collection.
        /// </summary>
        ~LocalUnmanagedMemory()
        {
            Dispose();
        }
        #endregion

        #region Methods
        #region Dispose (implementation of IDisposable)
        /// <summary>
        /// Releases the memory held by the <see cref="LocalUnmanagedMemory"/> object.
        /// </summary>
        public virtual void Dispose()
        {
            // Free the allocated memory
            Marshal.FreeHGlobal(Address);
            // Remove the pointer
            Address = IntPtr.Zero;
            // Avoid the finalizer
            GC.SuppressFinalize(this);
        }
        #endregion
        #region Read
        /// <summary>
        /// Reads data from the unmanaged block of memory.
        /// </summary>
        /// <typeparam name="T">The type of data to return.</typeparam>
        /// <returns>The return value is the block of memory casted in the specified type.</returns>
        public T Read<T>()
        {
            // Marshal data from the block of memory to a new allocated managed object
            return (T)Marshal.PtrToStructure(Address, typeof(T));
        }
        /// <summary>
        /// Reads an array of bytes from the unmanaged block of memory.
        /// </summary>
        /// <returns>The return value is the block of memory.</returns>
        public byte[] Read()
        {
            // Allocate an array to store data
            var bytes = new byte[Size];
            // Copy the block of memory to the array
            Marshal.Copy(Address, bytes, 0, Size);
            // Return the array
            return bytes;
        }
        #endregion
    }
}
\end{lstlisting}

\end{document}