\documentclass{article}

\usepackage[utf8]{inputenc}
\usepackage[T1]{fontenc}

% Required package to import.
\usepackage{minted}

% We can optionally uses a special theme to render the source code.
\usemintedstyle{pastie}

\begin{document}

\section*{An SQL query using \emph{minted}}

% Minted uses the environment: minted.
% Optional parameters can be passed between brackets. In this instance, the line numbers are displayed.
% Finally, the programming language is specified in curly brackets.
\begin{minted}[linenos]{sql}
CREATE OR REPLACE TRIGGER Tel_On_Off
AFTER INSERT ON STATE_CHANGE
FOR EACH ROW
WHEN (NEW.ChangeType='O' OR NEW.ChangeType='F')
DECLARE
  CELLA NUMBER;
  N_TEL_ATTIVI_MAX NUMBER;
BEGIN
  --Trovo l'ID della cella in cui si trova il mio cellulare
  SELECT CellID, MaxCalls INTO CELLA, N_TEL_ATTIVI_MAX
  FROM CELL
  WHERE x0<=:NEW.x AND x1>=:NEW.x AND
        y0<=:NEW.y AND y1>=:NEW.y;
        
  --Telefono acceso
  IF(:NEW.ChangeType='O') THEN
    --Inserisco il nuovo cellulare nella tabella
    INSERT INTO TELEPHONE(PhoneNo, x, y, PhoneState)
    VALUES(:NEW.PhoneNo, :NEW.x, :NEW.y, :NEW.ChangeType);
    --Aggiorno la cella in cui si trova il cellulare
    UPDATE CELL SET CurrentPhone#=CurrentPhone#+1
    WHERE CellID=CELLA;
  END IF;
  
  --Telefono spento
  IF(:NEW.ChangeType='F') THEN
    --Rimozione telefono da tebella
    DELETE FROM TELEPHONE WHERE PhoneNo=:NEW.PhoneNo;
    --Aggiorno la cella in cui si trova il cellulare
    UPDATE CELL SET CurrentPhone#=CurrentPhone#-1
    WHERE CellID=CELLA;
  END IF; 
END;
\end{minted}

\end{document}