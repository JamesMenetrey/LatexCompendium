\documentclass{article}

\usepackage[utf8]{inputenc}
\usepackage[T1]{fontenc}

% Required to use "m" and "b" options in the environment tabular
\usepackage{array}

\begin{document}

% The environment table renders the table as a floating element
\begin{table}
	\centering
	% The environment tabular defines the table. The parameters defines the number of columns.
	% The options in the parameter define the alignment of the content of the cells. They are:
	%	l			left-justified column
	%	c			centered column
	%	r			right-justified column
	%	p{width}	paragraph column with text vertically aligned at the top
	%	m{width}	paragraph column with text vertically aligned in the middle (requires array package)
	%	b{width}	paragraph column with text vertically aligned at the bottom (requires array package)
	%	|			vertical line
	%	||			double vertical line
	%
	% In the following example, three columns are separated by vertical borders.
	\begin{tabular}{| l | c | r |}
		% The command hline draws a horizontal border
		\hline
		000 & 000 & 000 \\
		\hline
		1 & 2 & 3 \\
		\hline
		4 & 5 & 6 \\
		\hline
		7 & 8 & 9 \\
		\hline
	\end{tabular}
	\caption{Basic table with borders}
\end{table}

\begin{table}
	\centering
	\begin{tabular}{| m{1cm} | p{2cm} | m{3cm} | m{4cm} |}
		\hline
		1cm 1cm 1cm 1cm 1cm & 2cm & 3cm & 4cm \\
		\hline
	\end{tabular}
	\caption{Basic table with fixed size of columns}
\end{table}

\end{document}